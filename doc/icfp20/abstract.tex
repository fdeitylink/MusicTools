Throughout history, music has been composed following general rules
and guidelines, expressed informally through natural language and
examples. The expressiveness of dependent type theory allows us to
capture these rules formally, and then use them to automate
analysis and synthesis of music.

In this experience report, we explore expressing a small subset of the
rules of the common practice period in Agda, a functional programming
language with full dependent types. We focus on the construction of
species counterpoint as well as four-part harmonization of melody. We
point out both the advantages of using dependent types to express
music theory and some of the challenges that remain to make
languages like Agda more practical as a tool for musical exploration.