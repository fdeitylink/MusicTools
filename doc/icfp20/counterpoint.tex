\section{Counterpoint}
\label{sec:cp}

Counterpoint is a technique for combining multiple lines of melodies.
Composing counterpoint is like arranging a song for a choir:
we start with a \emph{cantus firmus}, which serves as a base melody,
and compose counterpoint lines above or below the cantus firmus.
When doing this, we must make sure that the counterpoint lines sound
harmonically pleasing when played with the cantus firmus,
and that the individual melodic lines are distinguishable to the listener.

In this section, we show a dependently typed formalization of
two-voice counterpoint, more specifically,
those variants called the first and second species.
Here we restrict ourselves to \emph{upper} counterpoint
(i.e., composing a counterpoint line above the cantus firmus);
lower counterpoint can be formalized in the exactly same way.

\subsection{First Species Counterpoint}
\label{sec:cp:fs}

In the first species counterpoint, we start with a cantus firmus
consisting only of whole notes,
and set one note against each note in the cantus firmus.
To write good counterpoint, we must follow five rules explained below:

\paragraph{Beginning}
The beginning of the music must be a perfect consonance.
In the case of upper counterpoint, there are three ways to start the music:
using a unison, a 5th, or an 8th.

\paragraph{Intervals}
All intervals appearing in the main body must be consonant.
That is, the body should consist only of 3rds, 5ths, 6ths, and 8ths.

\paragraph{Motion}
For every pair of two adjacent intervals, the cantus firmus and counterpoint
lines must not move in the same direction if the second interval is a perfect one.
This is because such motion would draw too much attention
to the resulting interval.

\paragraph{Unisons}
Unisons should not occur in the main body of the music.
The reason is that they make it harder to distinguish between
the two melodic lines.

\paragraph{Ending}
The ending of the music must be a \emph{cadence}, a progression
of two intervals that induces a sense of resolution.
In our formalization, we fix the last interval to be an 8th,
hence the preceding interval should be either a major 6th or a minor 10th.
