\section{Counterpoint}
\label{sec:cp}

Counterpoint is a technique for combining multiple lines of melodies.
Composing counterpoint is like arranging a song for choir:
we start with a \emph{cantus firmus}, which serves as the base melody,
and compose counterpoint lines above or below the cantus firmus.
When doing this, we must make sure that the counterpoint lines sound
harmonically pleasing when played with the cantus firmus,
and that the individual melodic lines are distinguishable to the listener.

In this section, we present an implementation of species counterpoint,
based on the formulation given by Fux~\citep{fux-cp}.
The idea is to represent rules for composing ``good'' counterpoint
using dependent types.
[One more sentence here.]

\subsection{First Species Counterpoint}
\label{sec:cp:fs}

The simplest variant of counterpoint is called the first species
counterpoint.
In first species, we set one note against each note in the cantus
firmus, as in the following example:

[Frog]

In our implementation, we represent each bar as a pitch-interval
pair, and the whole music as a list of such pairs.

\begin{alltt}
PitchInterval : Set
PitchInterval = Pitch \(\times\) Interval
\end{alltt}

First species counterpoint comes with five mandatory rules, 
constraining the beginning, the main body, and the ending of
the music.
Below, we describe these rules one by one, showing the
corresponding Agda code.

\paragraph{Beginning}
The beginning of the music must be a perfect consonance,
that is, it must be a unison, a 5th, or an 8th.
In our implementation, this is checked by the predicate
\texttt{checkBeginning}, which is defined as follows:

\begin{alltt}
data BeginningError : Set where
  not158   : PitchInterval \(\rightarrow\) BeginningError
  tooShort : BeginningError
  
beginningCheck : PitchInterval \(\rightarrow\) Maybe BeginningError
beginningCheck pi@(_ , i) =
  if ((i == per1) \(\vee\) (i == per5) \(\vee\) (i == per8))
  then nothing
  else just (not158 pi)

checkBeginning : List PitchInterval \(\rightarrow\) Maybe BeginningError
checkBeginning []       = just tooShort
checkBeginning (p :: ps) = beginningCheck p  
\end{alltt}

\noindent The datatype \texttt{BeginningError} is a kind of
``type error'' designed specifically for counterpoint.
The first constructor \texttt{not158} tells us that the music
begins with a wrong interval, whereas the second constructor
\texttt{tooShort} says that the whole music is empty.
With this datatype, we can obtain an informative error message
when we accidentally break the rule at the beginning of the music.
Thus, what we are doing here can be seen as writing a
customized typechecker.

\paragraph{Intervals}
In the main body of the music, all intervals must be consonant.
That is, we should only use 3rds, 5ths, 6ths, and 8ths.
This rule can be encoded as the \texttt{checkIntervals} predicate,
which again uses a new datatype \texttt{IntervalError} representing
the presence of a dissonant interval:

\begin{alltt}
data IntervalError : Set where
  dissonant : Interval \(\rightarrow\) IntervalError

intervalCheck : Interval \(\rightarrow\) Maybe IntervalError
intervalCheck i = if isConsonant i then nothing else just (dissonant i)

checkIntervals : List PitchInterval \(\rightarrow\) List IntervalError
checkIntervals = mapMaybe (intervalCheck \(\circ\) proj\(\sb{\mathtt{2}}\))
\end{alltt}

\paragraph{Motion}
For every pair of two adjacent intervals, the cantus firmus and
counterpoint lines must not move in the same direction if the second
interval is a perfect one (i.e., unison, 5th, or 8th).
This is because such motion would draw too much attention to the
resulting interval.

\paragraph{Unisons}
Unisons should not occur in the main body of the music.
The reason is that they make it harder to distinguish between
the two melodic lines.

\paragraph{Ending}
The ending of the music must be a \emph{cadence}, a progression
of two intervals that gives rise to a sense of resolution.
