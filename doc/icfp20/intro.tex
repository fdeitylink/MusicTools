\section{Introduction}
\label{sec:intro}

Edgar Var\`{e}se describes music as ``organized sound'', and
throughout history cultures have developed and applied systems of rules
and guidelines to govern the music they create, most notably the
Common Practice Period of Western music spanning the 17th to early
20th centuries. These rule systems are seldom absolute, and indeed
deliberate breaking of the rules is often part of the aesthetic, but they
roughly constrain the music they apply to and
give it a common form and sound.

Artists and theoreticians have attempted to capture and codify these
rule systems, informally in natural language, and typically
accompanied by examples from the existing literature. The intent is
both to analyze existing music and then to use these principles to
guide the creation of new music, in other words for synthesis.

Starting in the 20th century, computers have become ubiquitous in
music in every area, including sound sythesis, composition and
production~\citep{roads-tutorial}. In terms of music theory, there has
been a line of recent work on using functional programming for
harmonic
analysis~\citep{magalhaes-harmtrace,dehaas-harmtrace-a,dehaas-harmtrace-b},
harmonization of a melody and generation of melodies based on a
harmonization~\citep{koops-fharm,magalhaes-fcomp} and
counterpoint~\citep{szamozvancev-welltyped}. There is also an
established Haskell library Euterpea~\citep{hudak-haskell} for general music
and sound exploration.

To describe rules of basic harmonic structure,
\citet{magalhaes-harmtrace} and their successors use dependent types,
for example to index chords by major or minor mode.  However Haskell
currently has limited support for dependent types, and requires many
extentions and tricks such as the use of singleton
types~\citep{eisenberg-singleton}.

In this paper we explore what can be done in the context of music by
using a programming language that offers full dependent types. We use
Agda~\citep{norell-phd} since it is fairly mature and
aims to be both a functional programming language and a proof
assistant. It also features a Haskell FFI so we can take advantage of
existing Haskell libraries, in partiular for MIDI generation.

% However this is difficult because most
% ``rules'' in music theory are not absolute, but rather suggestions of
% varying degrees (themselves vaguely specified) of importance. Since
% ultimately the resulting created music is more important than the
% theory, an alternative is to attempt to indirectly deduce the rules
% from music samples, using machine learning~\citep{huang-cp} or
% some other statistical methods. Unfortunately the set of deduced
% rules could sometimes be insufficient for generating good music.
% As an example, \citet{roberts-cp} develop the Bach Doodle
% harmonization tool using Bach chorals as training data, but still
% observe high frequency of parallel 5th and 8th that are extremely
% rare in Bach's music.

% Ultimately it seems some kind of mixture of rules and statistics may
% be the best way to faithfully capture an effective theory of
% music. One simple approach would be to develop a logic of rules and
% their interactions, but attach weights to the rules which could be
% determined via machine learning techniques.

Full dependent types allow expression of predicate logic, and it is
tempting to take a standard textbook on music theory such as
\citet{piston-harmony} or \citet{aldwell2018harmony} and formalize it
in type theory.  As a first step we start with the modest task of
expressing a small, relatively strict rule set known as species
counterpoint~\cite{fux-cp}, intended for combining interdependent
melody lines to produce a pleasant-sounding result.  Roughly speaking,
what we do is to write a custom musical type checker in Agda.
Compared to the previous work by \citet{cong-cp}, where rules are
expressed directly as Agda types, our approach makes it easier to
describe fine-grained rules, produce readable error messages, and add
or drop rules depending on the circumstances.

% [NOTE: I've squeezed the description of our new counterpoint
% implementation.]

% We make use of and extend Music Tools~\citep{MusicTools},
% a library of small tools
% for analysis and synthesis of music written in Agda, with the goal of
% eventually being a dependently-typed alternative to Euterpea.
% Previous work by \citet{cong-cp} expresses
% first-species counterpoint with the Agda type system, so that an error
% in counterpoint results in an Agda type error. Although this takes
% advantage of Agda's native type checking, it has several
% downsides. One is that the Agda errors may be difficult to interpret
% as the corresponding musical error. Another is that it is difficult to
% keep the types both simple and flexible.

% In this work we present an alternative representation of counterpoint
% using what could be considered a custom musical type checker written
% in Agda. This allows us to easily describe fine-grained rules, write
% custom type errors, and then combine subsets of rules together for
% larger-scale checking. Music can be combined with rules it
% satisfies as a dependent record type. This record represents a proof
% certificate that the music follows the asserted rules.
% Rules can also be used in multiple different contexts, satisfying
% modularity and composability.

We also present preliminary work on harmonizing a melody using
a subset of rules based on \citet{piston-harmony}, contrasting with
existing work by \citet{koops-fharm}. Since counterpoint and
harmony are not separate concepts but in fact deeply intertwined, we
would wish to reuse counterpoint rules to develop natural-sounding
harmonizations.  We show that, with our custom type checker, it is
straightforward to reuse rules in different contexts.  Put differently,
our representation of rules satisfies modularity.

The rest of this paper structured as follows.  Section~\ref{sec:music}
provides a brief glossary of musical terms, to prepare the
reader with no familiarity in music theory.  Sections~\ref{sec:cp} and
\ref{sec:harmony} describe our formalization of counterpoint and
harmony, focusing on the composable and modular aspects of the
proposed approach.  Lastly, Section~\ref{sec:conclusion} concludes
the paper with future perspectives.

This is an experience report, and we highlight both the advantages and
disadvantages of using Agda for music theory. On one hand the
expressiveness of dependent types makes it easy and natural to
describe music theory rules. On the other hand we find the emphasis on proof
construction and particularly the extra work needed for decidable
equality can add an extra burden which is not always welcome. However
overall we feel the positives far outweigh the negatives, and in the
final section we discuss possible ways to reduce the tedium.

