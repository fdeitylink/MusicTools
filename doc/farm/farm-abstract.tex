%% For double-blind review submission, w/o CCS and ACM Reference (max submission space)
\documentclass[sigplan,review]{acmart}\settopmatter{printfolios=true,printccs=false,printacmref=false}
%% For double-blind review submission, w/ CCS and ACM Reference
%\documentclass[acmsmall,review,anonymous]{acmart}\settopmatter{printfolios=true}
%% For single-blind review submission, w/o CCS and ACM Reference (max submission space)
%\documentclass[acmsmall,review]{acmart}\settopmatter{printfolios=true,printccs=false,printacmref=false}
%% For single-blind review submission, w/ CCS and ACM Reference
%\documentclass[acmsmall,review]{acmart}\settopmatter{printfolios=true}
%% For final camera-ready submission, w/ required CCS and ACM Reference
%\documentclass[acmsmall]{acmart}\settopmatter{}

%% Journal information
%% Supplied to authors by publisher for camera-ready submission;
%% use defaults for review submission.
\acmJournal{PACMPL}
\acmVolume{1}
\acmNumber{CONF} % CONF = POPL or ICFP or OOPSLA
\acmArticle{1}
\acmYear{2018}
\acmMonth{1}
\acmDOI{} % \acmDOI{10.1145/nnnnnnn.nnnnnnn}
\startPage{1}

%% Copyright information
%% Supplied to authors (based on authors' rights management selection;
%% see authors.acm.org) by publisher for camera-ready submission;
%% use 'none' for review submission.
\setcopyright{none}
%\setcopyright{acmcopyright}
%\setcopyright{acmlicensed}
%\setcopyright{rightsretained}
%\copyrightyear{2018}           %% If different from \acmYear

%% Bibliography style
\bibliographystyle{ACM-Reference-Format}
%% Citation style
%% Note: author/year citations are required for papers published as an
%% issue of PACMPL.
\citestyle{acmauthoryear}   %% For author/year citations

%% Some recommended packages.
\usepackage{booktabs}   %% For formal tables:
                        %% http://ctan.org/pkg/booktabs
\usepackage{subcaption} %% For complex figures with subfigures/subcaptions
                        %% http://ctan.org/pkg/subcaption


\begin{document}

%% Title information
\title{Demo: Counterpoint by Construction}
%\subtitle{Functional Pearl}

%% Author information
%% Contents and number of authors suppressed with 'anonymous'.

%% Author with single affiliation.
\author{Youyou Cong}
\affiliation{
  \institution{Tokyo Institute of Technology}            %% \institution is required
  \city{Tokyo}
  \country{Japan}
}
\email{cong@c.titech.ac.jp}          %% \email is recommended

\author{John Leo}
\affiliation{
  \institution{Halfaya Research}            %% \institution is required
  \city{Bellevue}
  \state{WA}
  \country{USA}
}
\email{leo@halfaya.org}          %% \email is recommended


%% Abstract
%% Note: \begin{abstract}...\end{abstract} environment must come
%% before \maketitle command


%% 2012 ACM Computing Classification System (CSS) concepts
%% Generate at 'http://dl.acm.org/ccs/ccs.cfm'.
\begin{CCSXML}
<ccs2012>
<concept>
<concept_id>10011007.10011006.10011008</concept_id>
<concept_desc>Software and its engineering~General programming languages</concept_desc>
<concept_significance>500</concept_significance>
</concept>
<concept>
<concept_id>10003456.10003457.10003521.10003525</concept_id>
<concept_desc>Social and professional topics~History of programming languages</concept_desc>
<concept_significance>300</concept_significance>
</concept>
</ccs2012>
\end{CCSXML}

\ccsdesc[500]{Software and its engineering~General programming languages}
\ccsdesc[300]{Social and professional topics~History of programming languages}
%% End of generated code

%% Keywords
%% comma separated list
% \keywords{keyword1, keyword2, keyword3}  %% \keywords are mandatory in final camera-ready submission


%% \maketitle
%% Note: \maketitle command must come after title commands, author
%% commands, abstract environment, Computing Classification System
%% environment and commands, and keywords command.
\maketitle

Western music of the common practice period tends to loosely
follow sets of rules, which were developed over time to ensure
the aethetic quality of the composition. 
Among various rules, those for harmony \citep{piston1987harmony}
and counterpoint \citep{fux1965study} (a technique for generating
multi-voice melodies) are continue to be taught to music students,
not only as a means to understand the music of that period, but
also as a foundation for modern art and popular music.

To help analyze and synthesize tonal music, it is worthwhile to
encode these rules into a programming language, and as shown by
recent studies, functional programming languages are particularly
suited to this task.
In the past decade, Haskell has been extensively used to encode
the rules of harmony \citep{fmmh, hihseufha, faamh, hafha, fghm}
as well as counterpoint \citep{szamozvancev2017well}.
Since Haskell is a statically typed language, these frameworks 
often come with a form of type safety property, namely
``well-typed music does not sound wrong''.
Unfortunately, it turns out that the type system of plain Haskell
is sometimes not powerful enough to guarantee this property.
To avoid generating music that ``sounds wrong'', previous studies
have incorporated various language extensions, such as GADTs 
\citep{cheney2002lightweight} and singleton types 
\citep{eisenberg2013dependently}, to enable dependently-typed 
programming.

In this demonstration of work in progress, we present Music Tools
\citep{MusicTools}, a library of small tools that can be combined
functionally to help analyze and synthesize music.
To allow simple and natural encoding of rules, we built the library
in Agda \citep{norellphd}, which is a functional language with full
dependent types.
As an application of the library, we demonstrate an implementation of 
species counterpoint, based on the rules given by \citet{fux1965study}.
Thanks to Agda's rich type system, we can ensure by construction that
well-typed counterpoint satisfies all the required rules.
We show how the type-based approach both aids human composition and
allows for computer-generated creation of correct counterpoint.
We also contrast our work to a recent study on generating
natural-sounding counterpoint by machine learning
\citep{CounterpointByConvolution}, which does not give us correctness
guarantees.
Finally, we discuss further applications of our library, including
handling of functional harmony.


%% Acknowledgments
\begin{acks}                            %% contents suppressed with 'anonymous'
 The authors would like to thank the participants of the Tokyo Agda 
 Implmentors' Meeting, especially Ulf Norell and Jesper Cockx,
 for many helpful suggestions that improved our Agda code.
\end{acks}


%% Bibliography
\bibliography{farm-abstract.bib}

\end{document}
